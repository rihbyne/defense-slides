\documentclass[9pt]{beamer}
%\usetheme[background=light]{metropolis}
\usetheme{Warsaw}
%\usefonttheme{professionalfonts}
\usefonttheme{serif}
\usepackage{fontspec}
%\usefonttheme{structurebold}
\usecolortheme{dove}
\setmainfont{Georgia}

\setbeamertemplate{headline}{%
\leavevmode%
  \hbox{%
    \begin{beamercolorbox}[wd=\paperwidth,ht=5.8ex,dp=2.125ex]{palette}%
        \insertsectionnavigationhorizontal{\paperwidth}{}{\hskip0pt plus1filll}
    \end{beamercolorbox}%
  }
}

\usepackage[utf8]{inputenc}
\usepackage{tikz}
\usetikzlibrary{shapes}

\setbeamertemplate{section in head/foot}{%
    \if\insertsectionheadnumber1
        \tikz\node[draw=black,fill=black,shape=signal,text=white]{\insertsectionhead\hskip.3cm};
    \else
        \tikz\node[draw=black,fill=black,shape=signal,signal from=west, signal to=east,text=white]     {\insertsectionhead\hskip.3cm};
  \fi
}

\setbeamertemplate{section in head/foot shaded}{%
    \if\insertsectionheadnumber1
        \tikz\node[draw=black,fill=white,shape=signal,text=black]{\insertsectionhead\hskip.3cm};
    \else
        \tikz\node[draw=black,fill=white,shape=signal,signal from=west, signal to=east,text=black]     {\insertsectionhead\hskip.3cm};
  \fi
}

\setbeamerfont{section in head/foot}{size=\fontsize{4}{6}\selectfont}

\title{Whirlpool}
\subtitle{Data Acquisition using N-node Distributed Web Crawler}
\author{Rihan Pereira, MSCS}
\institute[California State University Channel Islands]
{
%  %\inst{1}%
  \textit{Advisor:} Dr. Michael Soltys\\
  Department of Computer Science \\
  MSCS Graduate 2018-2019
}

\titlegraphic{\includegraphics[width=.3\textwidth,height=.2\textheight]{../csuci-mscs-thesis-dist-web-crawler/media/correctlogo.jpg}~}
\date{\today}

\begin{document}

% real meat of the presentation
% -------------------------------------
\begin{frame}[plain]
  \titlepage
\end{frame}

% table of contents a.k.a outline
% -------------------------------------
\begin{frame}[plain]
  %\frametitle{Agenda}
  \tableofcontents[hideallsubsections]
\end{frame}

\section[Motiv. \& Contrib]{Motivation \& Contribution}
\begin{frame}[plain]
  \centering{Motivation \& Contribution}
\end{frame}

% -------------------------------------

\begin{frame}{Motivation}
  \centering
  \includegraphics[width=8cm, height=6cm, keepaspectratio]{../csuci-mscs-thesis-dist-web-crawler/media/crawler/data_hierarchy.png}
  \pause
  \\
  Self-actualization (AI) is great, but you first need food, water, and shelter (data literacy, collection, and infrastructure).”
\end{frame}

% -------------------------------------

\begin{frame}{Contributions}
  to be completed
\end{frame}

% -------------------------------------

\section[Crawler history]{Crawler characteristics \& history}
\begin{frame}[plain]
  \centering{Crawler characteristics \& history}
\end{frame}

% -------------------------------------

\begin{frame}{Coverage \& Freshness}
  \centering
  \includegraphics[width=10cm, height=6cm, keepaspectratio]{../csuci-mscs-thesis-dist-web-crawler/media/crawler/crawl-ordering.png}
\end{frame}

% -------------------------------------

\begin{frame}{Web crawlers (1990 - 2019)}
  to add something
\end{frame}

% -------------------------------------

\section[Mercator]{Mercator 1999 (Heydon \& Najork)}
\begin{frame}[plain]
  \centering{Mercator 1999 (Heydon \& Najork)}
\end{frame}

% -------------------------------------

\begin{frame}{basic crawling algorithm}
  to add content
\end{frame}

% -------------------------------------

\begin{frame}{Mercator background}
  \centering
  \begin{figure}
  \includegraphics[width=10cm, height=6cm, keepaspectratio]{../csuci-mscs-thesis-dist-web-crawler/media/crawler/basic-crawler-architecture-v2.png}
  \caption{Mercator building blocks (Heydon \& Najork)}
  \end{figure}
\end{frame}

% -------------------------------------

\begin{frame}{URL Frontier Scheme}
  \centering
  \includegraphics[width=7cm, height=7.5cm, keepaspectratio]{../csuci-mscs-thesis-dist-web-crawler/media/crawler/url-frontier.png}
\end{frame}

% -------------------------------------

\begin{frame}{Front queue (Frontier Queue)}
  \centering
  \includegraphics[width=7cm, height=8cm, keepaspectratio]{../csuci-mscs-thesis-dist-web-crawler/media/crawler/f-queue.png}
\end{frame}

% -------------------------------------

\begin{frame}{Back queue (Frontier Queue)}
  \centering
  \includegraphics[width=8cm, height=9cm, keepaspectratio]{../csuci-mscs-thesis-dist-web-crawler/media/crawler/b-queue.png}
\end{frame}

% -------------------------------------

\section[Soft. design]{Software Design Principles}
\begin{frame}[plain]
  \centering{Software Design Principles}
\end{frame}

% -------------------------------------

\begin{frame}{Designing scalable systems}
  \begin{itemize}
    \pause
  \item Adding identical copies of components
    \pause
  \item Functional partitioning
    \begin{figure}
      \centering{\includegraphics[width=70mm, height=40mm, scale=0.1]{../csuci-mscs-thesis-dist-web-crawler/media/crawler/netflix.png}}
    \end{figure}
    \pause
  \item Data partitioning
  \end{itemize}
\end{frame}

% -------------------------------------

\begin{frame}{State Management}
  \only<1>{
    \centering{
      \begin{figure}
        \includegraphics[width=5cm, height=3cm, scale=1.3]{../csuci-mscs-thesis-dist-web-crawler/media/crawler/multi-cache-wrng.png}
        \caption{identical copies of same cached object}
      \end{figure}
    }
  }\only<2>{
    \centering{
      \begin{figure}
        \includegraphics[width=6cm, height=2cm, scale=0.1]{../csuci-mscs-thesis-dist-web-crawler/media/crawler/wrng-local-locks.png}
        \caption{Using local locks to access shared resources}
      \end{figure}
    }
  }\only<3>{
    \centering{
      \begin{figure}
        \includegraphics[width=6cm, height=2cm, scale=0.1]{../csuci-mscs-thesis-dist-web-crawler/media/crawler/rght-shared-locks.png}
        \caption{using shared locks to access shared resources}
      \end{figure}
    }
  }
\end{frame}

% -------------------------------------

\section[Event-driven]{Whirlpool: Event-driven architecture}
\begin{frame}[plain]
  \centering{Whirlpool: Event-driven architecture}
\end{frame}

% -------------------------------------

\begin{frame}{Message Queue(MQ)}
  \begin{columns}
    \column{0.7\textwidth}
    \centering{
      \includegraphics[width=8cm, height=6cm, keepaspectratio]{../csuci-mscs-thesis-dist-web-crawler/media/crawler/mq_basic.png}
    }
    \column{0.4\textwidth}
    \begin{itemize}
      \pause
    \item decoupling
      \pause
    \item scaled independently
      \pause
     \item balancing traffic
       \pause
     \item fault-tolerance
    \end{itemize}
  \end{columns}
  
\end{frame}

% -------------------------------------

\begin{frame}{MQ: Routing mechanisms}
  \begin{itemize}
    \pause
  \item Direct Worker Queue Data Flow
    \pause
  \item Fanout
    \pause
  \item Topic
    \pause
  \item Header
  \end{itemize}
\end{frame}

% -------------------------------------

\begin{frame}{Direct Worker Queue Data Flow}
  \centering
  \includegraphics[width=7cm, height=8cm, keepaspectratio]{../csuci-mscs-thesis-dist-web-crawler/media/crawler/rmq-broker.png}
\end{frame}

% -------------------------------------

\begin{frame}{RabbitMQ: Message bus}
  \centering
  \includegraphics[width=10cm, height=6cm, keepaspectratio]{../csuci-mscs-thesis-dist-web-crawler/media/crawler/multi-container-deploy.png}
\end{frame}

% -------------------------------------

\begin{frame}{development vs. production docker containers}
  things to add 
\end{frame}

% -------------------------------------

\section[Parser]{Whirlpool: Parser}
\begin{frame}[plain]
  \centering{Whirlpool: Parser}
\end{frame}

% -------------------------------------

\begin{frame}{Parser}
  to add something
\end{frame}

% -------------------------------------

\section[Deduplication]{Whirlpool: Near-Deduplication}
\begin{frame}[plain]
  \centering{Whirlpool: Near-Deduplication}
\end{frame}

% -------------------------------------

\begin{frame}{Dedupe}
  to add something
\end{frame}

% -------------------------------------

\section[Dist. Crawling]{Whirlpool: Distributed Crawling}
\begin{frame}[plain]
  \centering{Whirlpool: Distributed Crawling}
\end{frame}

% -------------------------------------

\begin{frame}{Dist. crawl}
  to add something
\end{frame}

% -------------------------------------

\section[Opworks]{Whirlpool: Operations}
\begin{frame}[plain]
  \centering{Whirlpool: Operations}
\end{frame}

% -------------------------------------

\subsection{From 10,000 ft.}
\begin{frame}{From 10,000 ft.}
 \centering
 \includegraphics[width=10cm, height=8cm, keepaspectratio]{../csuci-mscs-thesis-dist-web-crawler/media/crawler/ten-thousand-feet-aws.png} 
\end{frame}

% -------------------------------------

\subsection{From 5,000 ft.}
\begin{frame}{From 5,000 ft.}
  \centering
  \includegraphics[width=10cm, height=8cm, keepaspectratio]{../csuci-mscs-thesis-dist-web-crawler/media/crawler/aws-deploy-5k-feet.png}
\end{frame}

% -------------------------------------

\section[Future]{Future work}
\begin{frame}[plain]
  \centering{Future work}
\end{frame}

% -------------------------------------

\begin{frame}{future to do}
  to add something
\end{frame}

% -------------------------------------

%\section{Demo}
%\begin{frame}{Demo}
%  
%\end{frame}

% -------------------------------------

\begin{frame}[plain]
  \centering{Thank you! Questions ?}
\end{frame}
\end{document}
%%% Local Variables:
%%% mode: latex
%%% TeX-master: t
%%% End:
